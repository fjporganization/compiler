\documentclass{article}
\usepackage[czech]{babel}
\usepackage[utf8]{inputenc}
\usepackage[IL2]{fontenc}
\usepackage{graphicx}
\usepackage{amsmath}
\usepackage{hyperref}
\usepackage{array}
\usepackage{tabularx} 
\usepackage{longtable}
\usepackage{multirow}

\begin{document}
	
	\begin{longtable}{|l| p{10cm}|}
		\hline
		Instrukce & Popis \\ 
		\hline\hline
		
		\rule{0pt}{3ex} \texttt{LIT 0 M} & \textbf{Vložení hodnoty}: vloží literál \texttt{M} na vrchol zásobníku \\ \hline
		
		\rule{0pt}{3ex} \texttt{OPR 0 M} & Provádí aritmetické operace s celými čísly, konkrétní operace je udána hodnotou \texttt{M}. \\
		
		\rule{0pt}{4ex} \texttt{OPR 0 1} & \textbf{Unární mínus}: odebere hodnotu z vrcholu zásobníku a na vrchol zásobníku uloží její zápornou hodnotu. \\
		
		\rule{0pt}{4ex} \texttt{OPR 0 2} & \textbf{Sčítání}: odebere dvě hodnoty z vrcholu zásobníku a na vrchol zásobníku uloží jejich součet. \\
		
		\rule{0pt}{4ex} \texttt{OPR 0 3} & \textbf{Odčítání}: odebere dvě hodnoty z vrcholu zásobníku a na vrchol zásobníku uloží jejich rozdíl (první odebraná hodnota je menšitel, druhá odebraná hodnota je menšenec). \\
		
		\rule{0pt}{4ex} \texttt{OPR 0 4} & \textbf{Násobení}: odebere dvě hodnoty z vrcholu zásobníku a na vrchol zásobníku uloží jejich součin. \\
		
		\rule{0pt}{4ex} \texttt{OPR 0 5} & \textbf{Dělení}: odebere dvě hodnoty z vrcholu zásobníku a na vrchol zásobníku uloží jejich podíl (první odebraná hodnota je dělitel, druhá odebraná hodnota je dělenec). \\
		
		\rule{0pt}{4ex} \texttt{OPR 0 6} & \textbf{Modulo}: odebere dvě hodnoty z vrcholu zásobníku a na vrchol zásobníku uloží jejich zbytek po dělení (modulo). \\
		
		\rule{0pt}{4ex} \texttt{OPR 0 7} & \textbf{Určení sudosti}: odebere hodnotu z vrcholu zásobníku a na vrchol zásobníku uloží 1, pokud je hodnota sudá, jinak 0. \\
		
		\rule{0pt}{4ex} \texttt{OPR 0 8} & \textbf{Rovnost}: odebere dvě hodnoty z vrcholu zásobníku a na vrchol zásobníku uloží 1, pokud jsou si hodnoty rovny, jinak 0. \\
		
		\rule{0pt}{4ex} \texttt{OPR 0 9} & \textbf{Nerovnost}: odebere dvě hodnoty z vrcholu zásobníku a na vrchol zásobníku uloží 1, pokud jsou hodnoty rozdílné, jinak 0. \\
		
		\rule{0pt}{4ex} \texttt{OPR 0 10} & \textbf{Menší}: odebere dvě hodnoty z vrcholu zásobníku a na vrchol zásobníku uloží 1, pokud je druhá odebraná hodnota menší než první odebraná hodnota, jinak 0. \\
		
		\rule{0pt}{4ex} \texttt{OPR 0 11} & \textbf{Větší nebo rovno}: odebere dvě hodnoty z vrcholu zásobníku a na vrchol zásobníku uloží 1, pokud je druhá odebraná hodnota větší nebo rovna než první odebraná hodnota, jinak 0. \\
		
		\rule{0pt}{4ex} \texttt{OPR 0 12} & \textbf{Větší}: odebere dvě hodnoty z vrcholu zásobníku a na vrchol zásobníku uloží 1, pokud je druhá odebraná hodnota větší než první odebraná hodnota, jinak 0. \\
		
		\rule{0pt}{4ex} \texttt{OPR 0 13} & \textbf{Menší nebo rovno}: odebere dvě hodnoty z vrcholu zásobníku a na vrchol zásobníku uloží 1, pokud je druhá odebraná hodnota menší nebo rovna než první odebraná hodnota, jinak 0. \\ \hline
		
		\rule{0pt}{3ex} \texttt{LOD L M} & \textbf{Načtení}: načte hodnotu z vrcholu zásobníku na pozici \texttt{M} o \texttt{L} úrovní zanoření níže, načtenou hodnotu vloží na zásobník. \\
		\hline
		
		\rule{0pt}{3ex} \texttt{STO L M} & \textbf{Uložení}: uloží hodnotu z vrcholu zásobníku na pozici \texttt{M} o \texttt{L} úrovní zanoření níže. \\ \hline
		
		\rule{0pt}{3ex} \texttt{CAL L M} & \textbf{Volání procedury}: zavolá proceduru na pozici \texttt{M} o \texttt{L} úrovní zanoření níže. \\ \hline
		
		\rule{0pt}{3ex} \texttt{RET 0 0} & \textbf{Návrat z procedury}: vrací se z procedury do volající procedury. \\ \hline
		
		\rule{0pt}{3ex} \texttt{INT 0 M} & \textbf{Alokace na zásobníku}: alokuje na vrcholu zásobníku místo pro \texttt{M} hodnot. \\ \hline
		
		\rule{0pt}{3ex} \texttt{JMC 0 M} & \textbf{Podmíněný skok}: odebere z vrcholu zásobníku hodnotu a pokud je rovna nule, skočí na instrukci \texttt{M}. \\ \hline
		
		\rule{0pt}{3ex} \texttt{JMP 0 M} & \textbf{Skok}: skočí na instrukci \texttt{M}. \\ \hline
		
		\rule{0pt}{3ex} \texttt{REA 0 0} & \textbf{Načtení celého čísla}: načte celé číslo ze vstupu a uloží jej na zásobník. \\ \hline
		
		\rule{0pt}{3ex} \texttt{WRI 0 0} & \textbf{Výpis celého čísla}: odebere hodnotu z vrcholu zásobníku a vypíše ji na výstup. \\ \hline
		
		\rule{0pt}{3ex} \texttt{RER 0 0} & \textbf{Načtení reálného čísla}: načte reálné číslo ze vstupu a uloží jej na zásobník. Nejprve na zásobník vloží celou část čísla, poté vloží desetinnou část čísla. \\ \hline
		
		\rule{0pt}{3ex} \texttt{WRR 0 0} & \textbf{Výpis reálného čísla}: odebere nejprve desetinnou a poté celou část čísla ze zásobníku, následně desetinné číslo vypíše na výstup.\\ \hline
		
		\rule{0pt}{3ex} \texttt{OPF 0 M} & Provádí aritmetické operace s reálnými čísly, konkrétní operace je udána hodnotou \texttt{M}. Na zásobníku je reálné číslo rozděleno na dvě části, a to celou a desetinnou část; reálné číslo tedy zabírá dvě pozice v zásobníku.\\
		
		\rule{0pt}{4ex} \texttt{OPF 0 1} & \textbf{Unární mínus}: odebere dvě hodnoty z vrcholu zásobníku reprezentující reálné číslo a na vrchol zásobníku uloží zápornou hodnotu čísla, mínus je uloženo jen před celou část. \\
		
		\rule{0pt}{4ex} \texttt{OPF 0 2} & \textbf{Sčítání}: odebere čtyři hodnoty z vrcholu zásobníku představující dvě reálná čísla a na vrchol zásobníku uloží dvě hodnoty představující součet odebraných čísel \\
		
		\rule{0pt}{4ex} \texttt{OPF 0 3} & \textbf{Odčítání}: odebere čtyři hodnoty z vrcholu zásobníku představující dvě reálná čísla a na vrchol zásobníku uloží dvě hodnoty představující rodzíl odebraných čísel (první odebrané číslo je menšitel, druhé odebrané číslo je menšenec). \\
		
		\rule{0pt}{4ex} \texttt{OPF 0 4} & \textbf{Násobení}: odebere čtyři hodnoty z vrcholu zásobníku představující dvě reálná čísla a na vrchol zásobníku uloží dvě hodnoty představující součin odebraných čísel \\
		
		\rule{0pt}{4ex} \texttt{OPF 0 5} & \textbf{Dělení}: odebere čtyři hodnoty z vrcholu zásobníku představující dvě reálná čísla a na vrchol zásobníku uloží dvě hodnoty představující podíl odebraných čísel (první odebrané číslo je dělitel, druhé odebrané číslo je dělenec). \\		
		
		\rule{0pt}{4ex} \texttt{OPF 0 8} & \textbf{Rovnost}: odebere čtyři hodnoty z vrcholu zásobníku představující dvě reálná čísla a na vrchol zásobníku uloží 1, pokud jsou si čísla rovna, jinak 0. \\
		
		\rule{0pt}{4ex} \texttt{OPF 0 9} & \textbf{Nerovnost}: odebere čtyři hodnoty z vrcholu zásobníku představující dvě reálná čísla a na vrchol zásobníku uloží 1, pokud jsou čísla rozdílná, jinak 0. \\
		
		\rule{0pt}{4ex} \texttt{OPF 0 10} & \textbf{Menší}: odebere čtyři hodnoty z vrcholu zásobníku představující dvě reálná čísla a na vrchol zásobníku uloží 1, pokud je druhé odebrané číslo menší než první odebrané číslo, jinak 0. \\
		
		\rule{0pt}{4ex} \texttt{OPF 0 11} & \textbf{Větší nebo rovno}: odebere čtyři hodnoty z vrcholu zásobníku představující dvě reálná čísla a na vrchol zásobníku uloží 1, pokud je druhé odebrané číslo větší nebo rovno než první odebrané číslo, jinak 0. \\
		
		\rule{0pt}{4ex} \texttt{OPF 0 12} & \textbf{Větší}: odebere čtyři hodnoty z vrcholu zásobníku představující dvě reálná čísla a na vrchol zásobníku uloží 1, pokud je druhé odebrané číslo větší než první odebrané číslo, jinak 0. \\
		
		\rule{0pt}{4ex} \texttt{OPF 0 13} & \textbf{Menší nebo rovno}: odebere čtyři hodnoty z vrcholu zásobníku představující dvě reálná čísla a na vrchol zásobníku uloží 1, pokud je druhé odebrané číslo menší nebo rovno než první odebrané číslo, jinak 0. \\ \hline
		
		\rule{0pt}{3ex} \texttt{RTI 0 0} & \textbf{Reálné číslo na celé číslo}: odebere dvě hodnoty představující reálné číslo z vrcholu zásobníku a na vrchol zasobníku vloží pouze celou část reálného čísla. \\ \hline
		
		\rule{0pt}{3ex} \texttt{ITR 0 0} & \textbf{Celé číslo na reálné číslo}: odebere jednu hodnotu z vrcholu zásobníku a na vrchol zasobníku vloží dvě hodnoty představující reálné číslo. \\ \hline
		
		\rule{0pt}{3ex} \texttt{LDA 0 0} & \textbf{Načtení z absolutní adresy}: odebere z vrcholu zásobníku hodnotu představující absolutní adresu, na vrchol zásobníku vloží hodnotu nacházející se na odebrané absolutní adrese (bez ohledu na zanoření). \\ \hline
		
		\rule{0pt}{3ex} \texttt{STA 0 0} & \textbf{Uložení na absolutní adresu}: odebere ze zásobníku hodnotu představující absolutní pozici v zásobníku (bez ohledu na zanoření), na kterou uloží hodnotu odebranou ze zásobníku. \\ \hline
		
		\rule{0pt}{3ex} \texttt{REF 0 0} & \textbf{Načtení zlomku}: načte zlomek ze vstupu a uloží jej na zásobník. Na zásobník nejprve vloží čitatele, poté jmenovatele. \\ \hline
		
		\rule{0pt}{3ex} \texttt{WRF 0 0} & \textbf{Výpis zlomku}: odebere nejprve jmenovatele a poté čitatele zlomku, odebraný zlomek vypíše na výstup. \\ \hline
		
		\rule{0pt}{3ex} \texttt{NEW 0 0} & \textbf{Alokace na haldě}: alokuje jedno místo na haldě (prochází pole pro zásobník od konce a hledá nepoužívané místo), na zásobník poté vloží jednu hodnotu představující pozici alokovaného místa. \\ \hline
		
		\rule{0pt}{3ex} \texttt{DEL 0 0} & \textbf{Uvolnění místa na haldě}: odebere ze zásobníku jednu hodnotu představující adresu alokované proměnné, která se má uvolnit.\\ \hline
		
		\rule{0pt}{3ex} \texttt{PLD 0 0} & \textbf{Dynamické načtení}: odebere ze zásobníku dvě hodnoty, z nichž první odebraná představuje úroveň zanoření a druhá odebraná hodnota relativní pozici při dané úrovni zanoření, odkud se na vrchol zásobníku načte hodnota (odpovídá instrukci \texttt{LOD}, ale operandy jsou uloženy na zásobníku).\\ \hline
		
		\rule{0pt}{3ex} \texttt{PST 0 0} & \textbf{Dynamické uložení}: odebere ze zásobníku tři hodnoty, z nichž první odebraná představuje úroveň zanoření, druhá odebraná hodnota relativní pozici při dané úrovni zanoření a třetí odebraná hodnota představuje hodnotu k uložení (odpovídá instrukci \texttt{STO}, ale operandy jsou uloženy na zásobníku).\\ \hline
	\end{longtable}
\end{document}